%% start of file `moderncv_ntrp_template_en.tex'.
%% Copyright 2007 Xavier Danaux (xdanaux@gmail.com).
%
% This work may be distributed and/or modified under the
% conditions of the LaTeX Project Public License version 1.3c,
% available at http://www.latex-project.org/lppl/.
%
% Modded by ntrp (nitropowered@gmail.com)

\documentclass[11pt,a4paper]{moderncv}

% moderncv themes
%\moderncvtheme[blue]{casual}                 % optional argument are 'blue' (default), 'orange', 'red', 'green', 'grey' and 'roman' (for roman fonts, instead of sans serif fonts)
\moderncvtheme[blue]{classic}                % idem

\usepackage[T1]{fontenc}
% character encoding
\usepackage[utf8x]{inputenc}                   % replace by the encoding you are using

% adjust the page margins
\usepackage[scale=0.8]{geometry}
\recomputelengths                             % required when changes are made to page layout lengths

\fancyfoot{} % clear all footer fields
\fancyfoot[LE,RO]{\thepage}           % page number in "outer" position of footer line
\fancyfoot[RE,LO]{\footnotesize } % other info in "inner" position of footer line

% personal data
\firstname{Moisés}
\familyname{Ramírez Letelier}
\title{Curriculum Vitae}               % optional, remove the line if not wanted
\address{Av. Apoquindo 8200, dpto 48}    % optional, remove the line if not wanted
\mobile{+56 9 8577 6462}                    % optional, remove the line if not wanted
\email{source.moises@gmail.com}                      % optional, remove the line if not wanted
%\photo[84pt]{placeholder.jpg}                         % '64pt' is the height the picture must be resized to and 'picture' is the name of the picture file; optional, remove the line if not wanted

\nopagenumbers{}                             % uncomment to suppress automatic page numbering for CVs longer than one page


%----------------------------------------------------------------------------------
%            content
%----------------------------------------------------------------------------------
\begin{document}
\maketitle

%Section
\section{Información personal}
\cvline{Birth}{\small 21/01/1983 (Chile)\normalsize}
\cvline{Nacionalidad}{\small Chileno\normalsize}
\cvline{Github}{\small \url{https://github.com/moisesramirezl}\normalsize}
\cvline{LinkedIn}{\small \url{https://www.linkedin.com/in/mramirezl/}\normalsize}

%Section
\section{Resumen}
\cvline{}{Desarrollador fullstack con 13 años de experiecia}

%Section
\section{Habilidades técnicas} 
\cvcomputer{Lenguajes}{Javascript, Perl, Java, Python} {Otros}{	Redux, NodeJS, Typescript, Flow, CSS, HTML}
\cvcomputer{Monitoreo}{Grafana, Prometheus, Google Analytics, Splunk}  {Frameworks}{React, Django}
\cvcomputer{Infraestructura}{GCP, Akamai}  {Base de Datos}{MySql, MongoDB}
\cvcomputer{Herramientas}{VisualStudio, Atom, Itellij} {Metodologías} {Agile, XP, Scrum, Kanban}


%Section
\section{Experiencia}
\cventry{Enero 2020}{Ingeniero Principal de Software}{LATAM Airlines}{Santiago}{Chile}{
	Referente técnico de equipos ágiles que desarrollan productos digitales 
	para www.latam.com, responsable del ciclo de vida de los productos digitales del dominio 
	Order to Cash entre los que destacan: Flujo de compra de tickets y todo lo relacionado con el programa de pasajero frecuente.\\
	En lo concreto, en mi día a día estoy presente en:\\
	\begin{itemize}
			\item Construcción de las soluciones técnicas de los productos.
			\item Miembro de equipos temporales que diagnostican y solucionan incidentes complejos a nivel transversal.
			\item Velar por el cumplimiento de las buenas prácticas del desarrollo de software.
			\item Priorización de backlog, poniendo énfasis en el balance entre requerimientos técnicos y de negocio.
			\item Monitoreo de los productos a nivel operacional, de negocio y de usuario. Implementando monitores, reportes, alarmas y técnicas de SRE para explotar los datos que tenemos.
			\item Implementación del concepto de FCI (Failed Customer Interactions) para mejorar continuamente la experiencia de los clientes.
			\item Mentoring a desarrolladores.
			\item Miembro del equipo de revisión de código de aplicaciones críticas de sistemas legados.
			\item Participación en instancias de analisis de staff (organización del área, roles, partners, contrataciones, etc...)		   
	\end{itemize}
	} % arguments 3 to 6 are optional
\cventry{start-end}{<Position Held>}{<Name of employer>}{<Place>}{<Country>}{<Description>} % arguments 3 to 6 are optional

%Section
\section{Education}
\cventry{start-end}{<Title awarded>}{<Institution>}{<Place>}{<Country>}{<Description>} % arguments 3 to 6 are optional
\cventry{start-end}{<Title awarded>}{<Institution>}{<Place>}{<Country>}{<Description>} % arguments 3 to 6 are optional

%Section
%\section{Master thesis}
%\cvline{title}{\emph{Title}}
%\cvline{supervisors}{Supervisors}
%\cvline{description}{\small Short thesis abstract}

%Section
\section{languages}

\hspace{25mm}\small Self-assessment European level \href{http://europass.cedefop.europa.eu/en/resources/european-language-levels-cefr}{CEFR} (C2 maximum evaluation)\normalsize
\vspace{5mm}

\begin{tabular}{p{67mm} p{40mm} p{40mm} p{20mm}}
& \textbf{Understanding} & \textbf{Speaking} & \textbf{Writing} \\
\end{tabular}

\begin{tabular}{p{67mm} p{20mm} p{20mm} p{20mm} p{20mm} p{20mm}}
& Listening & Reading & Interaction & Production & \\
\end{tabular}

\vspace{3mm}
%lvl should be in this range A1 < A2 < B1 < B2 < C1 < C2
\cvlanguage{<Lang 1>}{<Level>}{
	\begin{tabular}{p{20mm} p{20mm} p{20mm} p{20mm} p{21mm}}
		lvl & lvl & lvl & lvl & lvl
	\end{tabular}}
\cvlanguage{<Lang 2>}{<Level>}{
	\begin{tabular}{p{20mm} p{20mm} p{20mm} p{20mm} p{21mm}}
		lvl & lvl & lvl & lvl & lvl
	\end{tabular}}

%Section
\section{Interests and Hobbies}
\cvline{}{\small <Description>}

%Section
\section{Extra}
\cvline{<Extra Content>}{\small <Description>}
\small
\cvlistitem{\href{...}{<Eventual link>}}
\cvlistitem{\href{...}{<Eventual link>}}

\closesection{}                   % needed to renewcommands
\renewcommand{\listitemsymbol}{-} % change the symbol for lists

% Publications from a BibTeX file
%\nocite{*}
%\bibliographystyle{plain}
%\bibliography{publications}       % 'publications' is the name of a BibTeX file

\end{document}