%% start of file `moderncv_ntrp_template_en.tex'.
%% Copyright 2007 Xavier Danaux (xdanaux@gmail.com).
%
% This work may be distributed and/or modified under the
% conditions of the LaTeX Project Public License version 1.3c,
% available at http://www.latex-project.org/lppl/.
%
% Modded by ntrp (nitropowered@gmail.com)

\documentclass[11pt,a4paper]{moderncv}

% moderncv themes
%\moderncvtheme[blue]{casual}                 % optional argument are 'blue' (default), 'orange', 'red', 'green', 'grey' and 'roman' (for roman fonts, instead of sans serif fonts)
\moderncvtheme[blue]{classic}                % idem

\usepackage[T1]{fontenc}
% character encoding
\usepackage[utf8x]{inputenc}                   % replace by the encoding you are using

% adjust the page margins
\usepackage[scale=0.8]{geometry}
\recomputelengths                             % required when changes are made to page layout lengths

\fancyfoot{} % clear all footer fields
\fancyfoot[LE,RO]{\thepage}           % page number in "outer" position of footer line
\fancyfoot[RE,LO]{\footnotesize } % other info in "inner" position of footer line

% personal data
\firstname{Moisés}
\familyname{Ramírez Letelier}
\title{Curriculum Vitae}               % optional, remove the line if not wanted
\address{Av. Apoquindo 8200, dpto 48}    % optional, remove the line if not wanted
\mobile{+56 9 8577 6462}                    % optional, remove the line if not wanted
\email{source.moises@gmail.com}                      % optional, remove the line if not wanted
%\photo[84pt]{placeholder.jpg}                         % '64pt' is the height the picture must be resized to and 'picture' is the name of the picture file; optional, remove the line if not wanted

%\nopagenumbers{}                             % uncomment to suppress automatic page numbering for CVs longer than one page


%----------------------------------------------------------------------------------
%            content
%----------------------------------------------------------------------------------
\begin{document}
\maketitle

%Section
\section{Información personal}
\cvline{Fecha de Nacimiento}{\small 21/01/1983 (Chile)\normalsize}
\cvline{Nacionalidad}{\small Chileno\normalsize}
\cvline{Github}{\small \url{https://github.com/moisesramirezl}\normalsize}
\cvline{LinkedIn}{\small \url{https://www.linkedin.com/in/mramirezl/}\normalsize}

%Section
\section{Resumen}
\cvline{}{\small Desarrollador fullstack con 13 años de experiecia}

%Section
\section{Educación}
\cventry{2001-2007}{Ingeniero Civil en Computación}{Universidad de Talca}{Chile}{}{Premio mejor alumno egresado promoción 2007}

%Section
\section{Habilidades técnicas}
\cvcomputer{Lenguajes}{Javascript, Perl, Java, Python} {Otros}{	Redux, NodeJS, Typescript, GIT, HTTP/REST, Flow, CSS, HTML, Groovy}
\cvcomputer{Monitoreo}{Grafana, Prometheus, Google Analytics, Splunk}  {Frameworks}{React, Django}
\cvcomputer{Infraestructura}{GCP, Akamai}  {Datos}{MySql, MongoDB, Microstrategy}
\cvcomputer{Herramientas}{Jenkins} {Metodologías} {Agile, XP, Scrum, Kanban}

%Section
\section{Habilidades Blandas}
Liderazgo técnico, comunicación, autodidacta, colaboración, investigación.

%Section
\section{Experiencia}
\cventry{ENE20}{Ingeniero Principal de Software}{LATAM Airlines}{Santiago}{Chile}{
	Referente técnico de equipos ágiles que desarrollan productos digitales
	para www.latam.com, responsable del ciclo de vida de los productos digitales del dominio
	Order to Cash entre los que destacan: Flujo de compra de tickets y todo lo relacionado con el programa de pasajero frecuente.\\
	En lo concreto, en mi día a día estoy presente en:\\
	\begin{itemize}
		\item Construcción de las soluciones técnicas de los productos.
		\item Miembro de equipos temporales que diagnostican y solucionan incidentes complejos a nivel transversal.
		\item Velar por el cumplimiento de las buenas prácticas del desarrollo de software.
		\item Priorización de backlog, poniendo énfasis en el balance entre requerimientos técnicos y de negocio.
		\item Monitoreo de los productos a nivel operacional, de negocio y de usuario. Implementando monitores, reportes, alarmas y técnicas de SRE para explotar los datos que tenemos.
		\item Implementación del concepto de FCI (Failed Customer Interactions) para mejorar continuamente la experiencia de los clientes.
		\item Mentoring a desarrolladores.
		\item Miembro del equipo de revisión de código de aplicaciones críticas de sistemas legados.
		\item Participación en instancias de analisis de staff (organización del área, roles, partners, contrataciones, etc...)
	\end{itemize}
	En paralelo participo como líder técnico en la iniciativa de Data Driven en un equipo en donde participan desarrolladores,
	C-Level manager y analistas de datos, que busca crear la cultura de la toma de decisiones en base a datos,
	entregando herramientas y acompañamiento en los equipos a nivel transversal para que puedan explotar sus datos.
}
\cventry{MAR16\\DIC19}{Ingeniero senior software}{LATAM Airlines}{Santiago}{Chile}{
	Líder técnico de equipo ágil que implementa productos digitales para el flujo de compra de tickets
	en www.latam.com. Este equipo estaba compuesto principalmente por desarrolladores de un partner que se encontraba en Brasil.\\
	Principales responsabilidades:
	\begin{itemize}
		\item Diseño e implementación de soluciones técnicas.
		\item Responsable del ciclo completo de las aplicaciones a cargo del equipo.
		\item Turno periodico 24/7 para incidentes críticos.
		\item Mentoring a ingenieros de software.
		\item Inducción, capacitación y acompañamiento a nuevos miembros del equipo.
		\item Facilitar el trabajo colaborativo en equipos remotos.
		\item Participación como representante del equipo en instancias con los stackeholders.
		\item Participación en comunidades internas
	\end{itemize}
	En paralelo participe en una iniciativa transversal a nivel LATAM sobre la construcción de un
	roadmap tecnológico de la compañía representando al área digital.\\
	Resultados principales:
	\begin{itemize}
		\item Implementación de un contenedor de aplicaciones genérico que se puede integrar en cualquier aplicación
		      y que permite agregar productos de manera dinámica para la venta en el canal web. Esto permitio que se pudiese
		\item Migración de aplicación legada de una arquitectura monolitica a una aplicación orientada a componentes front end
		      reutilizables y microservicios, aumentando la conversión del proceso de compra.
		\item Roadmap tecnológico del área digital
	\end{itemize}
}
\cventry{ENE12\\FEB16}{Jefe de Ingeniería de Software}{LATAM Airlines}{Santiago}{Chile}{
	Líder de equipo ágil que implementa productos digitales para el sitio www.latam.com.
	Principales responsabilidades:
	\begin{itemize}
		\item Seguimiento de plan de carrera de los ingenieros de software.
		\item Eliminar obstaculos e impedimentos que afectan a los integrantes del equipo.
		\item Participación en reuniones estrategicas y tácticas de la gerencia y comunicar y/o involucrar al equipo.
		\item Inducción, capacitación y acompañamiento a nuevos miembros del equipo.
		\item Mejorar las prácticas y herramientas del equipo/gerencia.
		\item Participación en la definición de estándares de desarrollo.
	\end{itemize}
	Resultados principales:
	\begin{itemize}
		\item Formación del equipo de mantenciones transversal con partner externo.
		\item Implementación de la nueva funcionalidad "Cambio de Fecha" para los clientes web.
	\end{itemize}
}
\cventry{NOV10\\FEB12}{Ingeniero Senior de Desarrollo}{LAN Airlines}{Santiago}{Chile}{
	Ingeniero de desarrollo de equipo ágil que implementa productos digitales para el sitio www.latam.com.
	Principales responsabilidades:
	\begin{itemize}
		\item Diseño e implementación de soluciones técnicas.
		\item Responsable del ciclo completo de las aplicaciones a cargo del equipo.
		\item Turno periodico 24/7 para incidentes críticos.
		\item Inducción, capacitación y acompañamiento a nuevos miembros del equipo.
		\item Participación como representante del equipo en instancias con los stackeholders.
	\end{itemize}
	Resultados principales:
	\begin{itemize}
		\item Implementación de robot para migrar reservas de aerolinea Aires a Lan.
	\end{itemize}
}
\cventry{ABR07\\NOV10}{Ingeniero de desarrollo}{LAN Airlines}{Santiago}{Chile}{
	Ingeniero de desarrollo que implementa productos digitales para el sitio www.latam.com.
	Principales responsabilidades:
	\begin{itemize}
		\item Diseño e implementación de soluciones técnicas.
		\item Responsable del ciclo completo de las aplicaciones a cargo del equipo.
		\item Turno periodico 24/7 para incidentes críticos.
		\item Captura de requerimientos con los stackeholders.
	\end{itemize}
	Resultados principales:
	\begin{itemize}
		\item Implementación de documento de comprobante de compra.
		\item Integración de medio de pago Ripley en el proceso de compra de tickets.
		\item Integración de medio de pago CMR en el proceso de compra de tickets.
	\end{itemize}}
\cventry{ENE05\\MAY05}{Desarrollador}{Viña Conosur}{Chimbrongo}{Chile}{
	Instalación y Programación de Estación Agrometeorológica y desarrollo de un sistema para de predicción de hongos en la vid.
}



%Section
\section{Lenguajes}
Español - Lengua Nativa\\
Ingles - Competencia profesional\\
Portugues - Basico\\

%Section
\section{Certificaciones}
\begin{itemize}
	\item Google Cloud Platform Big Data and Machine Learning Fundamentals
	\item Google Analytics For Beginners
	\item Advanced Google Analytics
	\item Splunk 7.x Fundamentals Part 1
	\item Google Cloud App Deployment, Debugging, and Performance
	\item Google Cloud Securing and Integrating Components of your Application
	\item Google Cloud Essential Cloud Infrastructure: Foundation
	\item Google Cloud Getting Started With Application Development
	\item Google Cloud Getting Started with Google Kubernetes Engine
	\item Google Cloud Platform Fundamentals: Core Infrastructure
	\item Certified ScrumMaster
	\item Certified ScrumDeveloper
\end{itemize}

%Section
\section{Intereses y Hobbies}
\cvline{}{\small Tecnología, Videojuegos, Música, Fútbol, Bicicleta}

%Section
\section{Referencias}
\cvline{<Extra Content>}{\small <Description>}
\small
\cvlistitem{\href{...}{<Eventual link>}}
\cvlistitem{\href{...}{<Eventual link>}}

\closesection{}                   % needed to renewcommands
\renewcommand{\listitemsymbol}{-} % change the symbol for lists

% Publications from a BibTeX file
%\nocite{*}
%\bibliographystyle{plain}
%\bibliography{publications}       % 'publications' is the name of a BibTeX file

\end{document}